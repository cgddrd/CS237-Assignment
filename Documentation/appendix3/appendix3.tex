\documentclass[12pt]{article}
\usepackage{listings}
\lstset{xleftmargin=-60pt,xrightmargin=-60pt}

\begin{document}


\section*{Appendix 3 - Main Mission File Load Log}

Original files can be found in: \textbf{Documentation/Original\_Logs/}.

\texttt{\lstinputlisting [breaklines=true, caption=Main Mission File Load Log (Minted)]{../Edited_Logs/main_load_logfile.txt}}

\end{document}